%==============================================================================
%------------------------------------------------------------------------------
%  Workshop handout for Git Internals workshop hosted by the Research Computing
%    Club at the University of Washington on May 10, 2019
%
%  Copyright (C) 2019 Andrew Wildman
%
%  The prose in this document is licensed under the Attribution-ShareAlike 4.0
%  International License (CC BY-SA 4.0). See LICENSE for more details.
%------------------------------------------------------------------------------
%==============================================================================

\documentclass[a4paper,12pt]{article}

\setlength\parindent{0pt}

\usepackage[utf8]{inputenc}
\usepackage[a4paper, total={7in, 10.5in}]{geometry}

\usepackage[T1]{fontenc}
\usepackage{libertine}%% Only as example for the romans/sans fonts
\usepackage[scaled=0.85]{beramono}

\begin{document}

\section*{Introduction}

\section*{Git object store}

\texttt{mkdir workshop\_repo\\
cd workshop\_repo\\
git init\\
ls .git}

\vspace{0.1in}

This is where we talk about the contents of the .git directory. As a reminder:
\begin{itemize}
  \item HEAD - Our location in the repo
  \item config - Repo specific information
  \item description - GitWeb specific file --- ignore
  \item hooks - Scripts that run on events (e.g. committing, pushing, etc.)
  \item info - Holds the exclude file, which is nice
  \item objects - Has two empty directories, but we'll work with it in a minute
  \item refs - Has two more empty directories, but we'll see it again
\end{itemize}

\vspace{0.1in}

\texttt{echo 'echo "Hello World!"' > hello.sh\\
git hash-object -w hello.sh\\
find .git/objects -type f}

\vspace{0.1in}

This is where we discuss the concept of a content-addressable filesystem.

\texttt{git hash-object -w hello.sh\\
find .git/objects -type f}

\vspace{0.1in}

Now we will get the contents of those files

\vspace{0.1in}

\texttt{git cat-file -p <hash>}

\vspace{0.1in}

And we can show how to restore files

\vspace{0.1in}

\texttt{rm hello.sh\\
git cat-file -p <hash> > hello.sh\\
cat hello.sh}

\vspace{0.1in}

Now let's make something a bit more complicated

\vspace{0.1in}

\texttt{mkdir src\\
echo '\#!/usr/bin/env python' > src/hello.py\\
echo 'print("Hello World")' >> src/hello.py\\
chmod +x src/hello.py\\
git hash-object src/hello.py\\
find .git/objects -type f}

\section*{Blobs and trees}

\texttt{git update-index --add hello.sh\\
ls .git\\
git ls-files\\
git status}

\vspace{0.1in}

Talk about permanent snapshots

\vspace{0.1in}

\texttt{git write-tree\\
find .git/objects -type f\\
git cat-file -p <hash>\\
git cat-file -t <hash>}

\vspace{0.1in}

Talk about how trees can point to blobs

\vspace{0.1in}

\texttt{git update-index --add src/hello.py\\
git ls-files\\
git write-tree\\
git cat-file -p <hash>\\
git cat-file -p <subdir hash>}

\vspace{0.1in}

Show tree structure

\section*{Commits}

\texttt{echo "Initial commit" | git commit-tree <hash>\\
git cat-file -t <hash>\\
find .git/objects -type f\\
git cat-file -p <hash>}

\vspace{0.1in}

Talk about the contents of a commit. Show commit structure slide.

\vspace{0.1in}

\texttt{mv src src\_moved\\
git update-index --add src\_moved/hello.py\\
git update-index --remove src/hello.py\\
git write-tree\\
echo "Moved src" | git commit-tree <tree hash> -p <other commit hash>\\
git cat-file -p <hash>}

\vspace{0.1in}

Talk about commit parents. Show new commit structure slide.

\vspace{0.1in}

\texttt{git log <commit hash>\\
find .git/objects -type f}

\vspace{0.1in}

Talk about blob persistence and show total repo structure slide.

Now let's make a merge commit.

\vspace{0.1in}

\texttt{mv src\_moved src\\
git update-index --add src/hello.py\\
git update-index --remove src\_moved/hello.py\\
mv hello.sh hi.sh\\
git update-index --add hi.sh\\
git update-index --remove hello.sh\\
git write-tree
echo "Renamed hello.sh" | git commit-tree <tree hash> -p <root hash>\\
mv src src\_moved\\
git update-index --add src\_moved/hello.py\\
git update-index --remove src/hello.py\\
git write-tree\\
echo "Merged histories" | git commit-tree <tree hash> -p <hash p1> -p <hash p2>\\
git log <hash>\\
git log <hash> --decorate --oneline --graph}

\vspace{0.1in}

Show the merge commit slide.

\section*{References}

\texttt{ls .git/refs\\
find .git/refs -type f\\
git update-ref refs/heads/master <newest hash>\\
find .git/refs -type f\\
cat .git/refs/heads/master}

\vspace{0.1in}

We can use this reference instead of a hash anywhere

\vspace{0.1in}

\texttt{git cat-file -p master\\
git log master\\
git rev-parse master}

\vspace{0.1in}

Now let's add another branch to show the branch we already merged

\vspace{0.1in}

\texttt{git update-ref refs/heads/dev <branch commit>\\
find .git/refs -type f\\
git branch}

\vspace{0.1in}

Talk about the special reference, HEAD

\vspace{0.1in}

\texttt{cat .git/HEAD}

\vspace{0.1in}

Now let's switch branches.

\vspace{0.1in}

\texttt{git symbolic-ref HEAD refs/heads/dev\\
git branch}

\vspace{0.1in}

Show the branches slide. What about the tags file?

\vspace{0.1in}

\texttt{git update-ref refs/tags/v0.0.1\_rc1 <moved src hash>\\
find .git/refs -type f\\
cat .git/refs/tags/v0.0.1\_rc1\\
git cat-file v0.0.1\_rc1}

\vspace{0.1in}

Talk about the lack of context for these tags

\vspace{0.1in}

\texttt{git tag -a v0.0.1 <merge hash> -m 'Released at the workshop'\\
git cat-file -p v0.0.1\\
find .git/objects -type f}

\vspace{0.1in}

Show the tags and branches slide.

\section*{Example: Rebase}
\begin{enumerate}
  \item Talk to the audience about the steps necessary to do a manual rebase
  \item Show them the source that can do it
\end{enumerate}

\section*{Audience choice}

Whatever they want to ask about - This is a possibility instead of the manual
rebase, especially if there is no time.

\end{document}
