%==============================================================================
%------------------------------------------------------------------------------
%  Workshop handout for Git Internals workshop hosted by the Research Computing
%    Club at the University of Washington on May 10, 2019
%
%  Copyright (C) 2019 Andrew Wildman
%
%  The prose in this document is licensed under the Attribution-ShareAlike 4.0
%  International License (CC BY-SA 4.0). See LICENSE for more details.
%------------------------------------------------------------------------------
%==============================================================================

\documentclass[a4paper]{article}

\usepackage[utf8]{inputenc}
\usepackage[a4paper, total={7in, 10.5in}]{geometry}

\usepackage[T1]{fontenc}
\usepackage{libertine}%% Only as example for the romans/sans fonts
\usepackage[scaled=0.85]{beramono}

\title{Git Internals Workshop}
\date{May 10, 2019}

\begin{document}
\maketitle

{\Large \texttt{$^*$NIX} Commands}

\begin{tabular}{ l l }
  \texttt{mkdir \textbf{<path>}} & Creates a directory/folder \\
  \hline
  \texttt{ls \textbf{<path>}} & Lists the contents of a directory \\
  \hline
  \texttt{find \textbf{<path>} -type f} & Finds all files in a given directory \\
  \hline
  \texttt{cat \textbf{<path>}} & Prints contents of a file to \texttt{stdout} \\
  \hline
  \texttt{echo \textbf{<message>} > \textbf{<path>}} &
    Overwrites a file with the given message \\
  \hline
\end{tabular}

\vspace{0.2in}

{\Large \texttt{git} Commands}

\begin{tabular}{ p{0.5\textwidth} p{0.4\textwidth} }
  \texttt{git init} & Creates and populates \texttt{.git/} directory \\
  \hline
  \texttt{git hash-object -w \textbf{<path>}} & Saves an object to the Git
    object store and prints the hash \\
  \hline
  \texttt{git cat-file -p \textbf{<object hash>}} & Prints the contents of an
    object in the Git object store \\
  \hline
  \texttt{git cat-file -t \textbf{<object hash>}} & Prints the type of an object
    in the Git object store \\
  \hline
  \texttt{git update-index -{}-add \textbf{<path>}} & Adds a file to the Git
    index \\
  \hline
  \texttt{git ls-files} & Lists the files in the current index \\
  \hline
  \texttt{git write-tree} & Saves the index as a tree in the Git object store
  and prints the hash \\
  \hline
  \texttt{echo \textbf{<message>} | git commit-tree \textbf{<tree hash>}} &
    Saves a commit pointing to a tree in the Git object store and prints the
    hash \\
  \hline
  \texttt{echo \textbf{<message>} | git commit-tree \textbf{<tree hash>} -p
  \textbf{<commit hash>}} &
    Saves a commit as above, but with (at least one) parent commit \\
  \hline
  \texttt{git log \textbf{<commit hash>}} & Prints all the commits previous to
    the one specified by the hash \\
  \hline
  \texttt{git log \textbf{<commit hash>} -{}-all -{}-decorate -{}-oneline -{}-graph} &
    Prints the log as a easily human parsable graph \\
  \hline
  \texttt{git update-ref refs/heads/\textbf{<branch>} \textbf{<commit hash>}} &
    Creates a named branched that points at the given commit \\
  \hline
  \texttt{git rev-parse \textbf{<branch>}} & Prints the commit hash to which the
    branch points \\
  \hline
  \texttt{git branch} & Lists all the branches in the repository \\
  \hline
  \texttt{git symbolic-ref HEAD refs/heads/\textbf{<branch>}} &
    Sets \texttt{HEAD} to point at the specified branch \\
  \hline
  \texttt{git update-ref refs/tags/\textbf{<tag>} \textbf{<object hash or reference>}} &
    Tags a git object with a lightweight tag \\
  \hline
  \texttt{git tag -a \textbf{<tag>} \textbf{<object hash or reference>} -m \textbf{<message>}} &
    Tags a git object with an annotated tag \\
  \hline
\end{tabular}

\vspace{0.5in}

{\Large Exercises}

\vspace{0.2in}

{\large Git Object Store}

\begin{enumerate}
  \item Create a new directory and initialize git
  \item Look inside the \texttt{.git} folder. Make sure you understand its
		  contents.
  \item Create a file and put some contents in it.
  \item Save that file to the git object store. Find it in the git object
		  database.
  \item Use a plumbing command to see the contents of the object you just saved.
		  \textit{Try steps 3-5 several times until you are comfortable saving
		  files to the database}
  \item Make a change to a file. Save the changed file to the object database.
		  Before you find it, what do you expect to see? Were you right?
  \item \textit{Bonus} Delete a file and recreate it from the git object store.
  \item \textit{Bonus} Change the name of a file, and save it to the object
		  database. Before you find it, what do you expect to see? Were you
		  right?
\end{enumerate}

\vspace{0.2in}

{\large Blobs and Trees}

\begin{enumerate}
  \item Add a file from part 1 to the index.
  \item Find the index file in the \texttt{git} folder.
  \item Check the contents of the index with a plumbing command. Confirm the
		  contents with \texttt{git status}.
  \item Save the index as a tree in the object database. Find the file in the
		  database.
  \item Look at the contents of the tree object. Make sure you understand each
		  line. \textit{Repeat 1-5 until you are comfortable saving trees to the
		  database.}
  \item Change a file and add it to the index. Save the tree to the object
		  database. Before you look at the contents of the tree, what do you
		  expect to see? Were you right?
  \item Save a subdirectory with files to your database. Inspect both the root
		  tree and the subtree. Make sure you understand each of the trees'
		  contents.
\end{enumerate}

\vspace{0.2in}

{\large Commits}

\begin{enumerate}
  \item Create a commit from a tree in part 2.
  \item Find your commit object in the object database.
  \item Check the contents of the commit you created. \textit{Repeat 1-3 to
          create a linear history}.
  \item Check the commit history with \texttt{git log}.
  \item \textit{Bonus} Create a second commit off your initial commit. Create a
		  merge commit from your two ''branches.``
\end{enumerate}

\vspace{0.2in}

{\large References}

\begin{enumerate}
  \item Look inside your \texttt{.git/refs} directory. Check that there are two
		  subdirectories.
  \item Take a commit from part 3 and create the ''master`` branch.
  \item Find the branch in the \texttt{.git/refs} directory.
  \item Create a new commit and the check that master has not changed. Advance
		  master to the newest commit.
  \item Create a second branch. Add some more commits, and advance this branch
		  to the newest commit.
  \item Create a lightweight tag on the most recent commit.
  \item Create an annotated tag on the same commit referred to by master.
  \item \textit{Bonus} Try to delete a branch using only file manipulation.
\end{enumerate}

\end{document}
